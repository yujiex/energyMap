\documentclass[12pt]{article}
\usepackage{hyperref}
\usepackage{enumitem,changepage,lipsum,titlesec}
\usepackage{cite}
\usepackage{comment, xcolor}
\usepackage[pdftex]{graphicx}
  \graphicspath{{images/}, {images/stat/}}
  \DeclareGraphicsExtensions{.pdf,.jpeg,.png, .jpg}
\usepackage[cmex10]{amsmath}
\usepackage{array} 
\usepackage[tight,footnotesize]{subfigure} 
\newcommand{\grey}[1]{\textcolor{black!30}{#1}}
\newcommand{\fref}[1]{Figure \ref{#1}}

\oddsidemargin0cm
\topmargin-2cm %I recommend adding these three lines to increase the
\textwidth16.5cm %amount of usable space on the page (and save trees)
\textheight23.5cm

\begin{document}
\title{Dynamic Energy Mapping Project Outline}
\maketitle
\tableofcontents
\newpage
\begin{abstract}
  This document provides an approach of adding the ``time'' dimension
  to an Energy Map. The approach is demonstrated with a model of a
  conceptual urban setting created in CityEngine based on the
  extracted topological and density pattern from an existing urban
  design project. The buildings in the conceptual model is then
  assigned an energy profile of certain DOE Commercial Benchmark
  Building Reference model based on its building type. Hourly energy
  demand profile of heating and cooling end use is then obtained from
  the EnergyPlus Reference models. The energy consumption data is
  classified into groups with consideration of building energy design
  context and the data distribution properties. A corresponding color
  coded energy profile is then generated and imported to
  CityEngine. 8760 color coded 3D map images was then extracted from
  CityEngine with Python script. A data reading, ploting,
  color-coding calculation and a user interface for visualizing the
  images and dynamic data plot with sliders is implemented using
  Python and related packages. The tool is anticipated to provide
  decision support for community energy management and
  planning, demand-side strategy design and district system sizing.
  
  The document will also briefly discuss one of the testbed for data
  classification and visualization.
\end{abstract}
\newpage
\section{General Introduction}
\subsection{Definition of Energy Map}
\subsubsection{Energy Thematic Map}
In a restricted sense, Energy Map is an instance of a thematic map
that depicts energy information. It is an abstract representation of
some energy feature in an urban environment. It is useful in providing
energy related qualitative or quantitative insight. 

The energy topics depicted in an Energy Map can be classified into
four major categories: energy supply, energy demand, energy related
building design / urban planning, and energy related environmental
impact. One common sub-category of the supply side topics concerns the
locations and evolving process of energy infrastructures such as power
plants, energy transmission pipelines, energy refining facilities and
market hubs. EIA state energy profile map~\cite{EIAProfile2015},
U.S. natural gas pipeline map~\cite{EIAGaspipe}, and U.S. wind farm
development dynamic map \cite{DOEWindFarm} are under this
sub-category. Other supply side topics include total energy production
~\cite{DOEEnergyProduct}; total energy source production like global
coal production map~\cite{EIACoalProduct}; sustainable energy
potential map of wind, solar, biomass, geothermal energy
potential~\cite{NRELMap2015} and hidropower
potential~\cite{DOEHydro}. Common demand side topics include: energy
demand for one or more enduses ~\cite{heatMap2012}, energy source
demand like coal demand ~\cite{EIACoalDemand} and energy cost
~\cite{DOEEnergyCost}. The design side topics concerns building
physical conditions like Calgory Heat Map ~\cite{Hay2011}, design
policy information like climate zone map~\cite{IECCClimate} and energy
code adoption map~\cite{CodeAdopt}. The energy behavior environmental
impact map include both the impact of building or energy
infrastructure to environment and the environment change to buildings
or infrastructures. The carbon emission map as ~\cite{CO2Atlas} is an
instance of the former and the ``Energy Sector's Vulnerabilities to
Climatic Conditions'' Map is an example of the
latter~\cite{DOEVulner}.

It is necessary to mention some unfortunate terminology overloading
involved in the topic of Energy Map. The term ``Heat Map'' used in
this discussion refers to the Energy Map with building heating energy
as its theme, not to be mis-interpreted as the color-coded
representation of matrix values as in this
definition~\cite{HeatmapWiki}.

The history of thematic map dates back to early 17th century, and from
then map can present spatial changes of some feature in addition to
merely recording locations of geographic
features~\cite{ThematicMap}. Over a century later, spatial analysis
emerges and map starts to assisting geo-data analysis. Finally
after the born of modern computer and the development of database, map
becomes a more powerful information system that undertakes more
complecated tasks including data aggregation, managing, querying and
presenting. This gives Energy Map a much broader meaning.

\subsubsection{Geo-database of Building Energy}
In a broader sense, Energy Map is a hibernation of two types of
databases: building energy database, a subset of the BIM (Building
Information Model), and Geographical Information System (GIS). The
basic functions of an Energy Map includes 1) storing energy data in an
organized fashion, that facilitate easy analysis and query of energy
data and 2) provide reports in the form of graphs, tables, animations
etc that conveys numerical information in a way that best support
pattern recognition and decision support. 

This definition can be considered as a superset of the thematic
definition, so the energy topics inherits those representable in the
thematic energy map. Some examples of the Energy Maps under this
definition include: National Heat Map that supports 

\subsection{Why ``time'' dimension is important for an Energy Map}
\subsubsection{Strong Temporal Variance Energy Demand}
Building energy demand is strongly dependent on weather conditions,
building types, size, building physical design, building mechanical
system and appliance selection and operation schedules. The factors of
building design, mechanical system and appliance selection remain
relatively constant over time. Weather conditions have strong seasonal
pattern and day-night pattern. This type of variation takes the form
of a global influence on building heating or cooling load. The
building operation schedule vary greatly from building to building as
a result of difference in building types, sizes, design, appliance
etc. Different operation scheduce indicates a non-coherent arrival of
peak demand within a mixed-use urban environment. Difference in
building type suggests a different indoor environment requirement such
as ventilation rate, lighting intencity etc., indicating a dramatic
variance in data distribution of energy consumption over
time. Building envolope and mechanical system quality indicate a great
variance in the range and extreme value of energy consumption. All of
these suggests a simple annual or monthly average cannot effectly
represent the real energy consumption behavior of an urban
environment. In order to present this complecated behavior of energy
demand, the time dimension is 

In order to match the supply side to the demand side,
understanding the spatial-temporal pattern of the energy demand is
crutial.
\subsubsection{Temporal variation of the Supply Side}
The commonly used renewable energy source includes: solar, wind,
geothermal, hydropower and biomass. Among these sources, solar energy
have strong temporal fluctuation as a result of the temporal variation
of solar radiation between different hours of a year and the time of
year~\cite{EIARenewable2015}. There is also a cost variation involved
in the energy supply.

\subsubsection{Close Match of Supply Side to Demand Side Improves Community Scale Energy Performance} 
As the result of the finiteness of fossil fuels, the using of
renewable energy begins to come into play. In 2013, renewable energy
account for 9\% of the residential and commercial primary energy
source~\cite{EIAPrimary2013}. Electricity generated from sustainable
sources normally do not have much storage capacity, hence in order to
meet the energy demand with renewable electricity, a better
understanding of the spatial-temporal pattern of energy demand is
important~\cite{Mikkola2014256}.

\subsubsection{Development of Supply Side Requires Better Understanding of Supply Side} 
\subsubsection{Community energy planning and district system design requires a more detailed picture of the energy temporal behavior on community level} 
\subsection{General Description of Dynamic Energy Map}
Within the current context, ``dynamic'' refers to changing over time,
hence Dynamic Energy Map is an Energy Map equipped with temporal
information. As a result of the ``dynamic'' property, one assumption
about Dynamic Energy Map is that at least one of the energy related
variables depicted in the map should change over time. Due to the fact
that there are two versions of definitions for Energy Map, there are
also two versions of corresponding Dynamic Energy Map.

\subsubsection{Thematic Map Time Series}
In a restricted sense, where an Energy Map is defined as a thematic
map focusing on energy topics, Dynamic Energy Map is just a series of
maps, each of which is a thematic Energy Map representing the status
of energy information happened at a certain time spot. Also with the
convention that thematic maps are ordered in increasing time order.
The purpose of such a Dynamic Energy Map is to facilitate the
comparison of thematic maps at different time steps. Baring this in
mind, it makes more sence to apply a universal map symbol and
breakpoints to the sequence of thematic maps in this version of
Dynamic Energy Map.

\subsubsection{Spatial-Temporal Energy-geo-database}
In a broader sence, where Energy Map is defined as a
energy-geo-database, Dynamic Energy Map is an energy-geo-database with
``time'' being one of its data entries. One major purpose of Dynamic
Energy Map under this definition is to enable search, filter and query
of the energy data by ``time'' field, thus presumably, time should act
as one of the indexes in the database that facilitates faster search
and query of the time data.  The second task of Dynamic Energy Map is
to provide more powerful reporting tools than normal Energy Maps that
accounts for the difficulty and complexity of spatial-temporal data
visualization aiming that better conveying the dynamic spatial
pattern.

In this study we explored the realization and application of Dynamic
Energy Map with a restricted use case senario of supporting district
energy system design and community energy management. 

A district energy system consists of a power plant, a series of
buildings as ternimal energy users and a network of pipelines that
transmit energy from the power plant to end-users. Commonly used media
for energy transmission include steam, hot water or chilled water~\cite{baird2014}.

\begin{comment}
For this scenario, we identify ``demand side'', 

with
heating and cooling demand as two major variables. 

\begin{itemize}
\item Dynamic Energy Map holds 8760-hour meta data of energy demand
  and supply~\cite{baird2014}.
\item Dynamic Energy Map has multi-dimensional graphical display of
  the meta data in conveying spatial-temporal pattern
  \begin{itemize}
  \item 1D: data plot for providing quantitative information
  \item 2D/3D: graphical display of spatial relationship of energy
    data
  \item 1D + 2D/3D: interactive graphical display of spatial-temporal
    pattern of energy data
  \end{itemize}
\end{itemize}

\newpage
\section{Objective}
%BookMark!!!!!
\begin{enumerate}[label*=\arabic*.]
\item Discuss the specifications / definitions of dynamic Energy Map
\item Evaluating some possible approaches to implement dynamic energy
  map
\item Presenting one major implementation approach
\end{enumerate}
\newpage
\section{Related Works}
\subsection{Energy Map (without temporal dimension) (grouped with field
  of application)}
\begin{enumerate}[label*=\arabic*.]
\item Supply side: Assessing renewable energy potential
  \begin{enumerate}[label*=\arabic*.]
  \item ``Evaluation of Renewable Energy potential using a {GIS}
    decision support system'', Voivontas et al., 1998
  \item ``Spatial mapping of renewable energy potential'', Ramachandra
    and Shruthi, 2007
  \item ``Energy Potential Mapping: Visualizing Energy
    Characteristics'', Dobbelsteen et al.\ , 2013
  \item ``NYC City Solar Map'': present solar energy potential for
    buildings across the city. Information presented include: solar
    energy generation curve, estimated solar system installation area,
    finantial incentive and payback etc\url{http://www.nycsolarmap.com/}.
    \end{enumerate}

\item Supply and Demand Side: Analysis or design support of existing energy infrastructures
    \begin{enumerate}[label*=\arabic*.]
    \item ``Developments to an existing city-wide district energy
    network – Part I: Identification of potential expansions using
    heat mapping'', Finney et al.\ , 2012
    \item National Heat Map, \url{http://tools.decc.gov.uk/nationalheatmap/}
    \end{enumerate}
\item Demand Side: Energy consumption prediction model
    \begin{enumerate}[label*=\arabic*.]
      \item ``A large-scale study on predicting and contextualizing building energy usage'', Kolter, J. Zico; Ferreira Jr, Joseph.
    \end{enumerate}
\item Smart Management of Urban Energy System
    \begin{enumerate}[label*=\arabic*.]
    \item ``Smart Urban Services for Higher Energy
    Efficiency''(SUNSHINE) project (2013): energy consumption map,
    automatic alerts, remote control of public building lighting
    system.
    \end{enumerate}
\end{enumerate}

  \item Dynamic Map
    \begin{enumerate}[label*=\arabic*.]
    \item History and Archaeology Instances of Dynamic Maps
      \begin{enumerate}[label*=\arabic*.]
      \item Pittsburgh Historic Map, \url{http://peoplemaps.esri.com/pittsburgh/} 
      \item Europe History Interactive Map, \url{http://www.worldology.com/Europe/europe\_history\_md.htm}
      \end{enumerate}
    \item Animated Maps
      \begin{enumerate}[label*=\arabic*.]
      \item ``The Role of Map Animation for Geographic
        Visualization'', Mark Harrower and Sara Fabrikant
      \item ``Using Computer Animation to Visualize Patterns'', D Dorling and S Openshaw, 1992
      \end{enumerate}
    \end{enumerate}
  \item Works on Visualization focusing on map design and information
    convey
    \begin{enumerate}[label*=\arabic*.]
    \item ``Data Visualization with Spacetime Maps'', Richard
      L. Brownrigg, 2005
    \item ``Effectiveness and efficiency of map symbols for dynamic
      geographic information visualization.'', Dong et al.\ 
    \item ``Geographic Visualization: Designing Manipulable Maps for
      Exploring Temporally Varying Georeferenced Statistics'', MacEachren et al.\
    \item ``Strategies for the Visualization of Geographic Time-Series
      Data'', Mark Monmonier, 2011
    \item ``Evaluation of Methods for Classifying Epidemiological Data
      on Choropleth Maps in Series'', Brewer and Pickle, 2002
    \end{enumerate}
  \item Works on Technology regarding 4D visualization
    \begin{enumerate}[label*=\arabic*.]
    \item ``Web-based 4D visualization of marine geo-data using
      WebGL'', Resch et al.\ , 2014
    \end{enumerate}
  \item Interface design of 4D visualization Case studies
    \begin{enumerate}[label*=\arabic*.]
    \item ``Web-based 4D visualization of marine geo-data using
      WebGL'', Resch et al.\ , 2014
    \end{enumerate}
  \end{enumerate}
\item Methodology
  \begin{enumerate}[label*=\arabic*.]
  \item General Work Flow
  \item Simulation Setting
    \begin{enumerate}[label*=\arabic*.]
    \item Source of benchmark models and default assumptions
    \item Modified settings: urban environment context
    \item Summary of input and output parameters
    \end{enumerate}
  \item Model Setting
    \begin{enumerate}[label*=\arabic*.]
    \item Software used in modeling and their general features (why
      choosing them)
    \item Process of extracting building layout from Mellon Arena
      Project
      \begin{enumerate}[label*=\arabic*.]
      \item Topological Pattern of the Mellon Arena Project
      \item Building Type converting
      \item Final Plan of the Conceptual Model
      \end{enumerate}
    \end{enumerate}
    \item Data Collection and Analysis
      \begin{enumerate}[label*=\arabic*.]
      \item Simulation Data Analysis of the benchmark models
        \begin{enumerate}[label*=\arabic*.]
        \item Distribution: Histogram, box plot
        \item Profile: Energy - Time plot
        \end{enumerate}
      \item Potential Impact on system design or data visualization
        based on the analysis above
      \end{enumerate}
    \item Temporal Data Aggregation
      \begin{enumerate}[label*=\arabic*.]
      \item With CityEngine
      \item With ArcGIS (ArcScene)
      \item Comparison
      \end{enumerate}
  \item Data Classification and symbol/color design of a dynamic
    choropleth map
    \begin{enumerate}[label*=\arabic*.]
    \item Review of General Approaches: see 3.3
    \item ``Critical Values'' or special cutoff values to be
      considered in the context of Community Energy Planning: need to
      look up (@@)
    \item Final choices of classification method and symbol/color
      scheme and the implementation
    \end{enumerate}
  \end{enumerate}
\item Interface Design
  \begin{enumerate}[label*=\arabic*.]
  \item Guidelines from interface design case study: See 2.5
  \item User definition
    \begin{enumerate}[label*=\arabic*.]
    \item Potential Users: policy makers, urban planners with the
      interest in executing community level energy strategies,
      researchers in energy related fields, public groups or
      individuals.
    \item Target users for the current project: researchers in energy
      related fields
      \begin{enumerate}[label*=\arabic*.]
      \item Assumptions about the skill level and background knowledge
      \item How the assumptions influence design choices of the
        interface design
      \end{enumerate}
    \end{enumerate}
  \item Goal Function of the interface
    \begin{enumerate}[label*=\arabic*.]
    \item Revealing the spatial-temporal heating / cooling demand
      variation of the conceptual model by applying the Dynamic
      Energy Map on a conceptual urban setting
    \end{enumerate}
  \item Major Operation Description
    \begin{enumerate}[label*=\arabic*.]
    \item Navigate through dynamic map images with time sliders
    \item Provide several default settings for choropleth map display
    \item Provide a brief help window and documentation of the tool
    \end{enumerate}
  \item Current Interface Design
    \begin{enumerate}[label*=\arabic*.]
    \item General Layout
    \item Navigation Function
      \begin{enumerate}[label*=\arabic*.]
      \item Overall navigation of year-round data
      \item Navigate and compare with default time steps: month, day,
        hour
      \end{enumerate}
    \item Dynamic Plot
    \item Implementation tools and strategy
    \end{enumerate}
  \end{enumerate}
\item Conclusion
  \begin{enumerate}[label*=\arabic*.]
  \item Summary of the current approach in implementing the dynamic
    Energy Map
  \item Limitations of the current implementation
    \begin{enumerate}[label*=\arabic*.]
    \item Simplified building simulation assumption about urban
      environment
    \item Lack of user choices for the stand-alone user interface as a
      result of its dependence on existing modeling softwares
    \end{enumerate}
  \item Future Expansion of the project
    \begin{enumerate}[label*=\arabic*.]
    \item Adding information of the supply side: residual energy,
      sustainable energy
    \item Providing different interfaces for different user population
    \item 2D and 3D compatible \\The reason for providing 2D map
      together with 3D map is that 2D maps have the following good
      properties:
      \begin{enumerate}[label*=\arabic*.]
      \item Better for region selection and spatial navigation than 3D
        map
      \item Better for conveying spatial relationship that does not
        involve height induced variation
      \item For larger scale display of city, state or nationwide, 3D
        building geometries becomes less significant in providing the
        urban environment context
      \end{enumerate}
    \item Creating an on-line tool for better information share
      \begin{enumerate}[label*=\arabic*.]
      \item Potential techniques: see 2.4
      \end{enumerate}
    \end{enumerate}
  \end{enumerate}
\item Acknowledgments
\end{enumerate}
\end{comment}
\newpage
\bibliographystyle{plain}
\bibliography{myCitation}
\end{document}