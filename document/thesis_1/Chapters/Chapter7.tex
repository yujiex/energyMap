% Chapter 7

\chapter{Findings and Discussion} % Main chapter title

\label{Chapter7} % For referencing the chapter elsewhere, use \ref{Chapter1} 

\lhead{Chapter 7. \emph{Findings and Discussion}} % This is for the header on each page - perhaps a shortened title

%----------------------------------------------------------------------------------------
"Findings and Discussion" section that describes your insights about
energy mapping, data needed, challenges of data display, issues of
showing the supply side, software limitations, etc. Your discussion of
limitations (6.2) can go here.  This seems like a short list of
limitations based on some of our discussions.  What about the
differences between mapping largely developed areas versus someplace
like Mellon Arena site?

\section{Insight about energy mapping}
\begin{itemize}
\item insights about energy mapping
\item data needed
\end{itemize}
\section{Limitations and Further Development}
\begin{itemize}
\item the break point is acquired with quantile classification method
  discussed in \sref{dataClassification}. Further analysis of suitable
  equipment sizing based break point would be added in future
  development of the project.

  \item Need supply side information and more detailed demand side
    information
  \begin{itemize}
  \item Renewable supply temporal information
  \item Energy quality (exergy) information of the supply and demand
    side
  \end{itemize}
\item Dependence on existing model
  \begin{itemize}
  \item User could not perform design and comparing the land use
    design on community energy planning because the current
    implementation relies on an existing modeling software for 3D
    image generation.
  \item Not fully control over data classification and legend selection
  \end{itemize}
\end{itemize}
