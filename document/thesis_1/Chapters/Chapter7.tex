% Chapter 7

\chapter{Findings and Discussion} % Main chapter title

\label{Chapter7} % For referencing the chapter elsewhere, use \ref{Chapter1} 

\lhead{Chapter 7. \emph{Findings and Discussion}} % This is for the header on each page - perhaps a shortened title

%----------------------------------------------------------------------------------------
\section{Insight of energy mapping with time information}
The document presents an approach of implementing a dynamic energy map
with a focus on visualization of high spatial-temporal energy demand
data of single buildings, building groups and the whole
community. 

In order to visualize the energy demand data to better convey the
dynamic energy demand changing, a detailed analysis of the input
demand data profile of each building type and the aggregated community
was conducted in \sref{boxPlot}. From this analysis, we observed a
great variation in the demand profile distribution between different
building types and a strong skew of the energy demand profile of the
whole community. Log scaling and data classification with the quantile
method was adopted to cope with this skewed data distribution in the
design of the conversion from energy demand to their color encoding.

An interface is designed to visualize the 2D/3D map images that
encodes energy demand information with a bivariate choropleth
legend. A series of data plot functions accompany the map image
display to provide quantitative information. The data plot provides
different level of temporal and spatial aggregation method that serve
more generic purpose of data analysis and visualization so that it
suits the need of the target user group: researchers in energy related
fields who have different research interests and focuses.

% bookmark
Through the dynamic energy map, many detailed spatial temporal energy
demand pattern can be revealed:
\begin{itemize}
\item The daily peak energy demand arrival time is different for
  different building types
\item The aggregation of peak demand is not a simple addition of peak
  demand of the peak of each component building's peak demand
\item The reject heat supply peak is different from the space heat
  demand peak
\item The heating and electricity demand for the community have very
  different weekly demand behavior
\end{itemize}
\section{Limitations and Further Development}
The current project presented a initial implementation of a dynamic
energy map with the focus on spatial-temporal energy profile
visualization. Due to limited time and resources, there are many to be
realized in the next stage development:

\begin{itemize}
\item Input data

  Community level energy simulation is not used in the energy profile
  generation. This leads to simplified assumptions of micro-climate in
  an urban environment. The building height and urban density could
  influence the exterior shading and could influence buildings'
  heating and cooling demand. The urban environment configurations of
  urban density and the spatial distribution of building height could
  potentially be conveyed with the dynamic energy map with 3D map
  sequences. However as a result of the stand-alone assumption, this
  layer of information is not demonstrated in the dynamic energy map
  in the current project.

  The current project focuses mainly on the demand side information,
  except that the cooling-induced reject heat is calculated. One of
  the major goal of energy mapping is to make the energy supply meet
  the energy demand. In order to fully realize the function of an
  energy map, adding supply side information is crutial. In the
  further development of the project, supply side information
  especially renewable energy supply assessment should be added to the
  dynamic energy map. This could pose new challenges of map design and
  data visualization as a result of the added layers of
  information. Further investment of proper spatial-temporal data
  visualization method could also become another topic of the next
  stage development of the project.

  The matching of energy demand and supply include not only matching
  in energy quantity but also energy quality in terms of exergy
  ~\cite{Dobbelsteen2013}. The current map interface does not contain
  exergy information. Further development could add exergy information
  in order to facilitate the design of energy cascading.

  When 

\item Visualization

  In the data classification step, the break point is acquired with
  quantile classification method as is discussed in
  \sref{dataClassification}. Further analysis of building science
  context based break point selection, such as critical value for
  mechanical equipment sizing or minimum recoverable heat considering
  transmission loss etc., should be taken into consideration in the
  next stage development of the project.
  
\item Data Sharing
  
  As is envisioned in the study of Baird et al.\ , an on-line platform
  is needed to facilitate data access, model sharing and advanced
  analysis~\cite{baird2014}. The current implementation of the dynamic
  energy map is a stand-alone tool. Bringing the current dynamic map
  to be an online map could be the next stage development.

\item Technical issue

  As a result of the dependence on an existing modeling software for
  image generation, user could not create community design and compare
  design alternatives on the current interface; user control over data
  classification and legend selection are not realized either. To
  solve this problem, a 3D model importing and rendering function
  should be added to the current map to facilitate the
  performance-based community energy planning.

\end{itemize}