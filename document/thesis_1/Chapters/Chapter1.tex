% Chapter 1

\chapter{General Introduction} % Main chapter title

\label{Chapter1} % For referencing the chapter elsewhere, use \ref{Chapter1}

\lhead{Chapter 1. \emph{General Introduction}} % This is for the header on each page - perhaps a shortened title

%----------------------------------------------------------------------------------------

\section{Project Overview}
The burning of fossil fuels produces green house gas (GHG) and causes
significant global climate changes including global sea level rise,
temperature rise, ocean warming, ice sheet melting and extreme weather
event~\cite{NASA2015}. Fossil fuels are finite: studies have shown
that if the consuming rate of fossil fuels remain the same, the major
fossil fuels including oil, gas and coal will run out by the end of
this century~\cite{Ecotricity2015, Kathryn2015}. Governments began to
put reducing GHG as one of their major development goals: UK launched
the ``Climate Change Act'' that aims at reducing GHG emissions by 80\%
comparing to 1990 by 2050~\cite{carbonBudgetUK}; the City of Calgary
aims at reducing CO$_2$ emission rate by 50\% by
2050~\cite{aacip2009}.

Reducing the GHG emission and the fossil fuel consumption also takes
place at the community level. Community Energy Management (CEM) is a
combination of community level design strategies and energy management
strategies aiming at providing quality of life in an urban environment
with minimized energy consumption and environmental
impact~\cite{Jaccard19971065}. It contains ``land use planning'',
``transportation management'', ``site planning'' and ``local energy
supply and delivery planning''~\cite{Jaccard19971065}. Community level
energy planning and management achieves GHG reduction by means of :1)
improving energy usage efficiency, 2) conserving the use of high
quality energy and 3) switching to using more renewable energy source
~\cite{StDenis20092088}.

Energy Mapping makes the community energy planning alternatives
visible to planners and policy makers~\cite{baird2014} and thus
flourishes with the increasing attention to community
planing. Emerging explorations on the role and power of Energy Mapping
in assisting community energy planning are taking place all over the
world. The City of Calgary carried out an Energy Mapping Study that
aims to ``encourage the use of alternative energy systems, through
considerations such as the design of buildings and encouragement of
more compact, mixed-use and high density
communities.''~\cite{baird2014}. The ``London Heat Map'' project that
helps developers and planners to ``identify opportunities for
decentralised energy projects''~\cite{londonHeatMap}.

What information should an Energy Map hold and in what form should the
information be conveyed or displayed is still not completely agreed
between different approaches. Calgary Energy Mapping study depicts
annual average energy use intensity and alternative renewable energy
supply region~\cite{aacip2009}. London Heat Map contains mainly
heating energy related features: high heating energy consumers,
suppliers and district heating networks. Dutch Heat Map, an
application of the Energy Potential Method (EPM) method developed by
Dobbelsteen et al.\ ~\cite{Dobbelsteen2013}, contains information of
annual heating energy demand (or demand density), heating energy
supply (or supply density), infrastructure network layout and CHP and
Biomass plant location.

However, as suggested by Baird et al.\ existing Energy Mapping
practices are mainly static, i.e. the time-dependent changes of energy
demand and supply information is not included in these Energy Maps nor
do they support more advanced community energy system analysis and
comparison. Thus the concept of ``Dynamic Energy
Mapping''~\cite{baird2014} was brought about:
\begin{enumerate}[i.]
\item It acts as a geo-database that efficiently holds
  \begin{itemize}
  \item hourly energy profile data for each building and the
    aggregated energy profile data for the whole
    community~\cite{baird2014};
  \item hourly energy supply data of community~\cite{baird2014}.
  \end{itemize}
\item It visualizes the dynamic energy demand and supply changes with
  high spatial and temporal resoltion~\cite{baird2014}.
\item It performs data analysis and supports district system
  sizing~\cite{baird2014}.
\item It can be connected to simulation tools that can supports
  instant performance analysis~\cite{baird2014}.
\end{enumerate}
 
With a Dynamic Energy Map, the temporal behavior of the demand and
supply of heating, cooling and electricity are revealed and are
available to be compared (function ii), analyzed (function iii) and
updated (function i and iv). One can see how well the supply meets the
demand over time. One can also use it as a key component of Geo-design
that encompasses ``geo-spatial modeling, impact simulations, and
real-time feedback to facilitate holistic designs and smart
decisions''~\cite{esriGeodesign2012}. The development of data-driven
approaches and machine learning methods could also be coupled and can
perform more complicated analysis of spatial-temporal behavior of
energy data and provide more informative design or management support.

\section{Objective and Problem Definition}
An initial instance of Dynamic Energy Map was created by Baird et al.\
in 2011-2012. The map consists of two parts: a geo-database that holds
general building information (name, conditioned area) annual and
monthly energy usage information (energy use intensity, annual peak
demand value and monthly peak demand value); an excel screening tool
that holds hourly energy usage information of each building and
performs analysis and system comparison of a district energy
system~\cite{baird2014}.

In the initial instance ~\cite{baird2014}, function i) of holding
spatial-temporal (although with low temporal resolution) energy data
is realized by processing the energy simulation data with Microsoft
excel and importing the csv file including ``building name, total
conditioned area annual, energy use intensity, annual and monthly peak
demand''. One goal of the current project is to make the geo-database
hold higher resolution energy data, i.e. the 8760 hourly energy data
of each building and the whole community will be contained in the
dynamic energy map.

Function iv) of connecting to building simulation data is also
realized by importing simulation result csv files to the geo-database
(although with low temporal resolution).

For function iii), the feasibility analysis of a district energy
system is performed in a stand-alone excel tool~\cite{baird2014} but
it is possible that the analysis result could be linked in to the
geo-database as the energy simulation result.

For function ii), the spatial and temporal information are visualized
separately in the initial instance~\cite{baird2014}: the spatial
information of 3D building geometry and location could be visually
inspected in the geo-database but not the hourly energy consumption
information. The temporal visualization of energy demand is done
separately in the excel screening tool as 3D graphs, but no spatial
context is present and the spatial dimension is then lost. 

The authors thus identified the crucially missing function: the
visualization of such a spatial-temporal changing of energy behavior
as the major goal of the current project.

The objective of the project is thus defined as to 
\begin{enumerate}
\item Implement a Dynamic Energy Demand Map with the focus on creating
  a high-resolution spatial-temporal visualization of hourly thermal
  energy consumption data for each building, major building sectors
  and the whole community
\item Demonstrate how such a Dynamic Energy Demand Map can support
  \begin{enumerate}
  \item Identification of energy recovery opportunities of single
    buildings or building groups
  \item Support the sizing of a district energy system CHP plant
  \end{enumerate}
\end{enumerate}

The community model is created in City Engine~\cite{cityEngine2015}
based on the land use pattern of a mixed-use redevelopment project at
Lower Hill District, Pittsburgh, PA~\cite{Ramesh2013}. The model
contains 68 buildings, comparable to a typical service area of a
district thermal energy system (combined heating and cooling), about
50 to 150 buildings~\cite{IDEA2005}.

The hourly heating cooling and electricity energy consumption profile
is retrieved from DOE Commercial Benchmark Building
simulation~\cite{DOE2015}. 

An interface was designed to combine the 8760 heating-cooling energy
choropleth map images from City Engine and the 8760 hourly
heating-cooling energy data from EnergyPlus to form a Dynamic Energy
Map. The interface provides users with the functions of 1) navigating
through the dynamic map images, 2) dynamic data plots of single
buildings, building sectors and aggregated community thermal energy
demand.

\section{Related Concepts}\label{concept}
Some related key concepts will be discussed in this section: the
district energy system, the Energy Map and the Dynamic Energy Map.

\subsection{District Energy System}
A district energy system is one form of Decentralized Energy System, a
``local or sub-regional supply of energy from a local
source.''~\cite{lhmreport2012}. It brings the energy generation near
to the energy end users and reduces the energy transmission and
distribution loss~\cite{decentralHeatMap2011}.

A district energy system produces thermal energy in a central plant
and delivers the thermal energy to local buildings through a
closed-loop pipeline network. Thermal energy are delivered in the form
of steam, hot water or chilled water~\cite{baird2014}. The central
power plant can take on one of the following forms: 1) thermal power
plant that only generates thermal energy, which can be either heating
or cooling energy 2) co-generation system, or combined heat and power
(CHP) system, that generates electricity and reuses the reject heat
from electricity generation to serve space heating and service hot
water to local buildings~\cite{IDEA2005} 3) tri-generation system,
where the central plant uses the heat generated by CHP plant to
produce chilled water and supply both heating and cooling energy
~\cite{cchp2015}. Corresponding to the different types of power plant,
the network delivering thermal energy can be classified as 1) district
heating network that only delivers steam or hot water 2) district
thermal network that delivers both heating energy in the form of steam
or hot water and cooling energy as chilled water.

A district thermal energy system corresponds to the three means of
community level GHG reduction as follows:
\begin{itemize}
\item It has high energy generation efficiency

  Higher energy usage efficiency means with the same amount of input
  energy, more useful energy is produced and less are
  wasted. Buildings' electricity supply are mainly from centralized
  power plant that are far away from cities. Heat produced in power
  generation are normally dumped into oceans and
  lakes~\cite{baird2014, IDEA2012}, not only causing negative
  environmental impact~\cite{wasteHeatEnviron}, but also reduce the
  energy generation efficiency to be only about
  1/3~\cite{IDEA2012}. District Energy System has high energy
  generation efficiency as a result of 1) it can utilize high
  efficiency large-scale energy generation equipment~\cite{IDEA2005}
  and 2) it is closer to the energy end user which reduces the energy
  loss due to transmission and distribution~\cite{IDEA2012}.

\item Better Exergy Performance 

  The quality of energy is usually described with exergy. It is
  defined as ``maximum useful work possible during a process that
  brings the system into equilibrium with a heat
  reservoir''~\cite{exergyWiki2015}. It represents the energy one can
  get out of the system. One example of a District Energy system helps
  improving exergy performance and better match the thermal energy
  supply and the low and medium-quality building energy demand
  ~\cite{Dobbelsteen2013} is the low-temperature (or low-energy)
  district heating system~\cite{Tol2012551} which has a supply
  temperature of around 50~$^o$C and return temperature of around
  25$^o$~C~\cite{Tol2012551}.

\item Multiple fuel choices including renewable energy sources

  The central plant of a district energy system can use a broad range
  of fuel choices including natural gas, oil, coal, waste, and
  renewable energy sources including geothermal, solar thermal and
  biomass, in the generation of thermal energy. This makes the switch
  to large scale renewable energy source possible. It also makes the
  district thermal energy system more flexible and more competitive in
  the market and increases the energy system
  resilience~\cite{IDEA2005, IDEA2012}.

\end{itemize}

Apart from the environmental benefits, a district energy system also
reduces the space and cost dedicated to installation and maintenance
of HVAC systems in single buildings. It also reduces harmful gas
emission of NO$_x$, SO$_x$ by using non-combustion energy sources as
lake body and by filtering~\cite{IDEA2012} the flue
gas~\cite{veolia2014}.

\subsection{Heat Map}
Although ``heat map'' is generally accepted as ``graphical
representation of data where the individual values contained in a
matrix are represented as colors''~\cite{HeatmapWiki}, with respect to
buildings, a ``heat map'' may be defined as ``a spatial plan of
existing and planned building heat demand, and decentralized energy
networks and generation equipment''~\cite{decentralHeatMap2011}. It is
also a GIS ``live database'' that allows new development information
to be incorporated. It is a key component to the decentralized energy
master plan~\cite{decentralHeatMap2011}. One of the well-known
instances of a heat map is the ``London Heat
Map''~\cite{londonHeatMap}

\subsection{Energy Map}
International District Energy Association (IDEA) define Energy Map as:
``a tool that can be used to organize/present data as the basis for
defining energy character areas as part of energy
planning''~\cite{IDEA2012}. It is a ``GIS based system'' that can be
used to develop energy strategies, prioritize project, identify
potential growth opportunities and impose planning
restrictions~\cite{IDEA2012}. Dobbelsteen et al.\ adopted the term
``Energy Potential Mapping (EPM)''~\cite{Dobbelsteen2013}. EPM assists
the development and plan of a sustainable built environment. It is a
method that ``visualizes local energy potentials and demand in order
to support spatial planning towards more energy-efficient urban or
rural environment''~\cite{Dobbelsteen2013}. UK used the
``Decentralized Energy Masterplanning'' as a method that helps local
authorities identify low carbon strategies that ``maximises the
opportunity for large-scale schemes to capture and use waste heat from
major energy sources''~\cite{decentralHeatMap2011}.

With respects to the various definitions above, an Energy Map could be
understood as a generalization of a heat map that includes energy
supply, demand and infrastructure information of various energy forms
and technologies. Some existing use cases suggest an Energy Map could
be used to visualize the community or city level energy demand
reduction with high performance building design~\cite{aacip2009} or
adoption of alternative energy supply
technologies~\cite{aacip2009}. It can be used in supporting district
heating system design~\cite{decentralHeatMap2011, Finney2012165} by
visualizing the heat sources and sinks.

\subsection{Dynamic Energy Map}
According to the study of Baird et al.\ , a Dynamic Energy Map is an
Energy Map equipped with temporal information of energy supply and
demand. It enables spatial-temporal comparison, aggregation and query
of energy demand and supply. It is coupled with Energy simulation
tools, and design alternatives would be evaluated and compared at each
given time spot or time period. By performing advanced data analysis
method, the dynamic map makes patterns that are omitted in static maps
visible and analyzable. Both aspects enable more detailed energy
analysis and design support.

\section{Why ``time'' dimension is important}
\subsection{Strong Temporal Variation of Energy Demand}
Different building types often indicates different energy demand
profile. For example, the residential building heat demand profile has
two major peaks, morning and evening, and is relatively low for the
rest of the day. For office buildings, there is a peak heat demand in
the morning and a relatively high heat demand through the day time but
drops in the evening. Hospitals usually have a more flattened demand
throughout the day. Within a mixed-used urban environment, the arrival
of peak demand for different buildings are usually not
simultaneous~\cite{decentralHeatMap2011}.

In the design of a district energy system, mixing building types with
different time-of-use energy profile can be helpful in creating a less
variate aggregated energy demand. This allows the central CHP plant in
a district energy system to a have higher utilization rate and reduces
the need for backup plant that accounts for high peak
demand~\cite{decentralHeatMap2011}.

\begin{figure}[h!]
  \centering
  \includegraphics[width=0.7\linewidth]{mixLoad.png}
  \caption[Mixing Load Graph]{Mixing Load of Different Building
    Type~\cite{decentralHeatMap2011}}
  \label{fig:mixLoad}
\end{figure}

\subsection{Aggregation of Peak Value Becomes Tricky for Data with
  Time Variation}
One common mistake for sizing a district thermal energy system is to
add up the peak demand of each terminal users. Since the peak demand
of individual buildings do not occur at the same time, the end result
of summing up the peak demand at each end point exceeds the actual
total demand peak of the community. Hence with this approach, the
whole district system becomes excessively over-sized, which reduces
the whole system efficiency. A Dynamic Energy Map can reveal the
problem of such approaches by directly providing the aggregated
thermal energy and electricity demand for single buildings, building
sectors or the whole community. It allows a side by side comparison of
single building demand and aggregated demand and eliminates the
misunderstanding of the demand aggregation. With the direct
information of aggregated thermal energy and electricity demand, it
also assists actually sizing a district thermal energy system.



