% Chapter 5

\chapter{Literature Study on Map Design and Data
  Visualization} % Main chapter title

\label{Chapter5} % For referencing the chapter elsewhere, use \ref{Chapter1} 

\lhead{Chapter 5. \emph{Map Design}} % This is for the header on each page - perhaps a shortened title

%----------------------------------------------------------------------------------------
\subsection{Animated Maps}\label{anime}
Animated maps are proven to be more powerful in conveying the
spatial-temporal pattern than static maps~\cite{McEachern1998}.

In order to represent dynamic geographic processes, map animation is a
natural choice. It was introduced to the world of cartography in
1930s~\cite{Harrower2008}. The major application of animated maps
include: 1) demonstrating the dynamic process of geographic events
(weather maps in weather forecasting is such an example) 2) assisting
pattern recognition and knowledge development for scientific
researches. The study by Dorling and Openshaw is an example of 
application 2), where they discovered new leukaemia hotspots through
animated maps~\cite{Dorling1992}.

Animated maps are not superior to static maps, it is just they are
good at different aspects of information convey. The animated map is
advantageous in demonstrating the changes between frames rather than
the absolute value represented in each frame~\cite{Dorling1992}. It is
proved to be more powerful in convey the spatial-temporal pattern than
static map~\cite{McEachern1998}.

Harrower and Fabrikant mentioned that the chanllenge of using animated
maps is the overflow of information and the vulnerability to
distraction~\cite{Harrower2008}. One example mentioned by Harrower and
Fabrikant is the comparison of color on the map and that on the legend
becomes difficult for animated maps as a result of the changing of
images. They proposed the audio legend approach of strengthening
information convey with minimized
distraction~\cite{Harrower2008}. This might become one of the next
extensions of the current Dynamic Energy Map interface design.

They also suggested that the difference in time should have different
visual representations in data display~\cite{Harrower2008}. Peuquet
claimed that ``The development of temporal analytical capabilities in
GIS such as temporal queries requires basic topological structures in
both time and space''~\cite{Peuquet1994}. Thus the different spatial
representation seems to be a natural choice for adapting to different
temporal resolution and scale.

They classify time into two types: linear and cyclic. Upon this
consideration, the design of the current interface include both an
overall time navigation utility and time navigation utilities that
facilitate jumps with time steps corresponding to the natural period
of energy data, such as month, day and hour. This design choice
is anticipated to facilitate the representation of both linear changes
and periodical changes of energy usage in the community.

The level of user control of playback behavior of animated maps is
also debatable. Some claim providing the full freedom of adjusting
this feature can enhance pattern understanding~\cite{Nelson1998}. But
others argue that this control will reduce time animation to still
images and impair its ability in conveying temporal changes
~\cite{Lowe2004}. In the current dynamic map project, both the
interactive version and the non-interactive version is provided: the
non-interactive version (a map animation) is accessible through
\href{http://www.armechxyj.com/energy-mapping.html#redblueAnime3d}{here}. The
interactive version is provided as a stand-alone Python program.

\subsection{Spatial Temporal Data Analysis}\label{stDataAnalysis}
In order to better utilize the power of Dynamic Maps, one has to
understand the special features of spactial-temporal data and the
methods of how to use spactial-temporal data. This leads to the
literature study of the following section of spatial temporal data
analysis.

One temptation of analyzing spatial-temporal data is to aggregate them
into ``time periods'' and ``zonal entities'' and then use the static
analysis method to analyze the aggregated data~\cite{Dorling1992}. The
problem of this approach is 1) it increases the sensitivity
(i.e. minor changes in input causes dramatic changes in output) and 2)
it removes the ``dynamic'' feature of a dynamic.
map~\cite{Dorling1992}.

One layer of the goal of a space-time map is to make ``complex dynamic
process'' visible, in the hope of letting observers comprehend the
dynamics of data presented and to gain a general insight. Baring this
goal in mind, Dorling and Openshaw suggested a noise removal or data
smoothing in both the time and space dimension before the actual map
creation~\cite{Dorling1992}.

\section{Data Classification}\label{dataClassification}
In order to write the data classification routine for the
demonstration of dynamic energy map in the current study, the authors
conducted a brief survey of the commonly used GIS software for
commonly applied data classification method. The software surveyed in
the study include: ArcGIS~\cite{GIS_Jenks2014}, GRASS
GIS~\cite{GRASSGIS2008}, gvGIS~\cite{gvGIS2011}, and QGIS. The data
classification method adopted by the surveyed software in creating a
thematic map include: 1) equal interval, 2) quantile 3) Jenks 4)
Standard Deviation 5) pretty breaks 6) manual interval (use context
specific break point values). The common data classification method
shared by all surveyed instances are ``Equal Interval'', ``Quantile''
and ``User Defined''. Therefore we chose to implement the ``Equal
Interval'' and ``Quantile'' method in the current project.

\begin{table}[h!]
  \centering
  \begin{tabular}{r|c c c c c c}
    \hline
           & Equal Interval & Quantile & Jenks & Pretty Breaks & StDev & User Defined\\
    \hline
    ArcGIS &      o        &    o     &  o    & x &  o  &   o  \\
 GRASS GIS &      o        &    o     &  x    & x &  o  &   o  \\
     GVSIG &      o        &    o     &  o    & x &  x  &   o  \\
      QGIS &      o        &    o     &  o    & o &  o  &   o  \\
    \hline
  \end{tabular}
  \caption{Data Classification Method (o: yes, x: no)}
  \label{tab:classify}
\end{table}

\begin{comment}

``Data Visualization with Spacetime Maps'', Richard L. Brownrigg, 2005
(read further later on)

\grey{To be continued later:
\begin{enumerate}[label*=\arabic*.]
\item ``Geographic Visualization: Designing Manipulable Maps for
    Exploring Temporally Varying Georeferenced Statistics'', MacEachren et al.\
\item ``Strategies for the Visualization of Geographic Time-Series
    Data'', Mark Monmonier, 2011
\item ``Evaluation of Methods for Classifying Epidemiological Data
    on Choropleth Maps in Series'', Brewer and Pickle, 2002
\end{enumerate}}
\end{comment}



However, the level of user control of playback behavior is debatable
according to previous literature about animated maps. Some claim
providing the full freedom of adjusting this feature can enhance
pattern understanding~\cite{Nelson1998}. But others argue that this
control will reduce time animation to still images and impair its
ability in conveying temporal changes ~\cite{Lowe2004}.